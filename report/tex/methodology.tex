\subsection{Research design}
In order to replicate my result, please do the following steps:

\begin{enumerate}
    \item Access the original dataset provided by Juha Nurmi at \url{https://
    mega.nz/folder/aJwVyIYJ#}.
    \item Download \emph{database.tar.gz} (7 MB) and follow the instructions
    in \emph{README.md}.
    \item Check \emph{sha256sum}:\\\textbf{5a5f2cb4feb7fee597b0a26b1dc2fb33b1f
    9cae639e995a89663198bcfa76f1a}.
    \item Extract the compressed dataset (tar -xf database.tar.gz).
    \item 33,896 JSON files are generated in the destination folder.
    \item Clone the GitHub repo \url{https://github.com/ancuongnguyen07/Database
    _Market.git} and follow the comprehensive \emph{README.md} document.
\end{enumerate}

%%%%% DATA SAMPLE
\subsection{Samples}
%
The Database Market that sells personal data is accessible through \url{http://d
atabase6e2t4yvdsrbw3qq6votzyfzspaso7sjga2tchx6tov23nsid.onion/} inside the
anonymous Tor network. The initial dataset, provided by teacher Juha Nurmi, can
be downloaded from \url{https://mega.nz/folder/aJwVyIYJ#9SWh-Z3-
TpPfjHZeFxbeew}.
The provided dataset is an archive of 33,896 JSON files, 400 MB of raw size. Each
JSON file is represented as the following fields as shown in \autoref{fig:original_json}:

\begin{itemize}
    \item \emph{url}: URL address of webpage.
    \item \emph{text}: Text of the webpage.
    \item \emph{timestamp}: Data collection date.
\end{itemize}

\begin{figure}
    \centering
    \includegraphics[width=\textwidth,height=\textheight,keepaspectratio]
    {screenshots/orginal_json.png}
    \caption{An example JSON file from the provided dataset by Juha.}\label{fig:original_json}
\end{figure}

In order to achieve more insights of credentials for sale, I produced an additional JSON file
,\emph{ProductPages},that reveals more key features of products sold on a targeted marketplace. Each entry in
the customized JSON file represents a webpage of product containing many sorts of items for sale.
\autoref{fig:custom_json} is an example of an entry, where I assigned the following fields.

\begin{itemize}
    \item \emph{id}: Product ID\@.
    \item \emph{time-stamp}: Data collection date.
    \item \emph{category}: Type of product.
    \item \emph{seller}: Username of seller.
    \item \emph{product}: Name of product.
    \item \emph{prices}: Prices of items.
    \item \emph{dates}: Item uploaded date.
\end{itemize}

\begin{figure}
    \centering
    \includegraphics[width=\textwidth,height=\textheight,keepaspectratio]
    {screenshots/customized_json.png}
    \caption{A example JSON entry from my customized dataset}\label{fig:custom_json}
\end{figure}

The datasets are under CC BY 4.0 license: You are free to copy, share, redistribute,
remix, transform, and build upon the material for any purpose, even commercially.
You must give attribution and appropriate credit. Follow the terms: \url{https:/
creativecommons.org/licenses/by/4.0/}.

%%%%% Data collection
\subsection{Data collection}
%
The initial dataset captured web pages of stolen credentials sold in the Database market
from November 2021 to June 2022. Based on that dataset, I extracted a certain of key
fields, including \emph{product, prices, dates} and \emph{seller} for the \emph{text} field.
In addition, I manually took screenshots of pages showing information of products and
stores in Database Market. In order to access a marketplace hosted in Tor network,
it is common that you need an invitation code to register an account on the marketplace.
However, in case of Database Market, I can create an account without an invitation
code.

Filtering interesting fields from the text content in the webpage of product requiring
some marking letters. For example, I noticed that the title of item sold in Database
marketplace is enclosed by ``\#\#\#\#\#\#'', and the store name is covered by ``\#\#\#''. However,
it is more complicated in the cases of \emph{prices} and \emph{dates}. As criminals can
unconstrainedly format the description of their products, I am unable to conduct a common
pattern to capture all prices and dates of uploaded items. Thus, in addition to the quite
effective pattern, returned correct fields in most of pages, I added some criteria for
specific cases. In terms of category, I filtered top common keywords in the title of
products, and grouped items that contain the shared keyword into a category.

%%%% Data analysis
\subsection{Data analysis}
%
Applying statistic to the whole dataset, \emph{ProductPages}, I obtained the following
numerical results.

\begin{table}
    \centering
    \begin{tabular}{|c|c|}
        \hline
        Total number of items & 53815\\
        Total number of stores & 30\\
        Maximum price & 2500.00 USD\\
        Minimum price & 0.20 USD\\
        Average price & 15.87 USD\\
        Median price & 5.00 USD\\
        \hline
    \end{tabular}
    \caption{Statistical result of the \emph{ProductPages} dataset}
    \label{tab:dataset_stat}
\end{table}